\documentclass[]{article}

%opening
\title{Compiti di matematica}
\author{Liceo Einstein Torino}
\usepackage[a4paper, margin=1in]{geometry}
\setlength{\parindent}{0cm}
\thispagestyle{empty}
\pagestyle{empty}
\newcommand*\vf {\hspace*{0em plus 1fill}\makebox{V \quad F}}
\newcommand*\result[1] {\hspace*{0em plus 1fill}{\footnotesize \makebox{[#1]}}}
\usepackage{graphicx,amsfonts,eurosym}
\graphicspath{ {images/} }
\usepackage[shortlabels]{enumitem}

\begin{document}

{\vspace*{-2.5cm}\hspace*{15.5cm}\includegraphics[height=2.5cm]{logo}}

\begin{center}
	{\vspace*{-0.5cm}\LARGE\bfseries Esercizi di matematica per le future prime\par}
	\par{di tutti gli indirizzi del liceo scientifico}
\end{center}

\parskip=10pt

\par{Gli esercizi seguenti risulteranno utili se i calcoli saranno eseguiti mentalmente, applicando le proprietà delle quattro operazioni (commutativa, associativa e distributiva.)}

\parskip=10pt

\par{\textit{Esempio: }\(452 + 128 + 437 = 400 + 100 + 400 + 50 + 20 + 30 + 2 + 8 + 7 = 1000 + 17 = 1017\)}

\parskip=20pt

\begin{enumerate}
	\item \quad \(2 + 192 + 1728 + 342 = \)
	\item \quad \(72 + 2936 + 77002 + 25 = \)
	\item \quad \(547 - 52 = 547 - 50 - 2\)
	\item \quad \(1237 - 120 = \)
	\item \quad \(2348 - 1028 = \)
	\item \quad \(235 \times 36 = 235 \times 30 + 235 \times 6 = \)
	\item \quad \(8738 \times 107 = \)
	\item \quad \(1,23 \times 0,06 = \)
	\item {Di quanto aumenta un numero di due cifre, avente la cifra delle decine uguale a 5, se si inserisce uno zero tra la cifra delle decine e quella delle unità?}
	\item \quad \(347 - 4 \times \{ 25 - 7 \times [ 18 - 12 \times \, 3 \times 4 - 11\, - 2 \times 3 ] \} \) \result{247}
	\item \quad \(\{ [ \, 0,05 + 0,2 \times 0,7 \, \times 0,4 + 0,8 \times 0,03] \times 0,3 + 0,5 \} \times 2 - 1 \) \result{0,06}
	\item {Se si raddoppia ciascuno dei due fattori di una moltiplicazione, come varia il prodotto?}
	\item {Dire se le seguenti proposizioni sono vere o false:
	\begin{itemize}
		\item {il prodotto di due numeri decimali, maggiori di 1, è sempre maggiore di 1} \vf
		\item {il prodotto di un numero decimale minore di 1 per un numero decimale \\ maggiore di 1 è minore o uguale a 1} \vf
		\item {il prodotto di due numeri decimali minori di 1 è minore di 1} \vf
		\item {se un prodotto è nullo, almeno uno dei due fattori è nullo} \vf
		\item {il prodotto di due numeri decimali è maggiore o uguale a ciascuno dei due fattori} \vf
	\end{itemize}
	}

	\item \quad \( \{ 3 + 196 : [7+7\times\,15:3-2\,]-\,8\times9-12\,:12 \}:[\,9\times9+12\,:3-78:3] \) \result{1}
	
	\item \quad \(\,0,3+2\times0,6\,:\{[\,1,3\times2,4-2,375\,:0,05+\,1,1]\times0,03+\,5\times6+1,2:0,3\,\times0,03\}+9\) \result{10}
	
	\item {Di quanto si deve diminuire il lato di un quadrato, lungo 118 cm, per far diminuire l'area di 2043 cm\(^2\)} \result{9 cm}
	
	\item {Da un foglio di carta quadrata la cui area è 10404 cm\(^2\), si vogliono ritagliare dei quadratini, aventi ciascuno il lato lungo 4 cm. Calcolare il massimo numero di quadratini che si possono ottenere.} \result{650}
	
	\item {In una cassa a forma di cubo con lo spigolo lungo 1,65 m si ripongono dei cubi di legno aventi ciascuno spigolo pari a 12 cm. Quanti cubi contiene la cassa? Qual è il volume della parte di cassa non utilizzata?} \result{2197; 696 cm\(^3\)}
	
	\item {Risolvi le seguenti espressioni numeriche in \(\mathbb{N}\) applicando, quando possibile, le proprietà delle potenze:
		\begin{enumerate}[a.]
			\item \([4^2\times(4\times4^3)^2:4^6]^2:\,4^3\times4^2\) \result{\(4^3\)}
			\item \([(3^4\times3^5)^2:3^{10}]^2:\{3^{10}:[(3^8:3^5):3]^3\}^2\) \result{\(3^8\)}
			\item \([(5^2\times5)^4:5^9]^4\times[(5^4:5^2)^3:(5^2\times5^3)]^0\) \result{\(5^12\)}
			\item \({[2^7:(2^8:2^5)^2]^{10}:2^4}^5:[(2^3\times2^2\times2)^3:2^8]^3\) \result{1}
			\item \(\{[(7^3)^2:(7^5:7^2)]^2:(7^4\times7)\}^3\times[(7\times7^0)^0\times7^2]^2\) \result{\(7^7\)}
			\item \(\{[(3^2)^3\times3^2]^2:(3\times3^3)^4\}^7\times[(3^2\times3^3)^2:3^5]^2\) \result{\(3^{10}\)}
			\item \(100^3:\{[2^2+(5^2:5^2+5):26]^2:3+(18^3:6^3)-2^2\}^3\) \result{8}
		\end{enumerate}
	}

	\item {La notazione scientifica è un metodo per esprimere un numero quale prodotto di un numero compreso tra 1 e 10 e una potenza di 10 \\ \\ \textit{Esempio: } \(72400 = 7,24 \times 10^4\) \\ \\ Scrivere in notazione scientifica i seguenti numeri:
	\begin{itemize}
			\item 497100 =
			\item 32000000 =
			\item 120000000 =
			\item 3290000000 =
			\item 48000 =
			\item 780000 =
			\item 0,00045 =
			\item 0,0014 =
			\item 0,0000000011 = 
	\end{itemize}
	}

	\item {Scrivere i seguenti numeri nella notazione usuale:
	\begin{itemize}
		\item \(3,2 \times 10^3\) =
		\item \(1,24 \times 10^5\) =
		\item \(1,002 \times 10^7\) =
		\item \(4,7 \times 10^{-7}\) =
		\item \(1,25 \times 10^{-2}\) =
		\item \(1,2 \times 10^{-6}\) =
	\end{itemize}	
	}
	
	\item {Stabilire se i seguenti numeri sono divisibili per 2, 3, 4, 5, 9, 10, 11, 25, 100:
		\begin{itemize}
			\item 2232
			\item 6072
			\item 12312
			\item 17820
			\item 24480
		\end{itemize}
	}
	
	\item {Dire se le seguenti proposizioni sono vere o false:
	\begin{itemize}
		\item {un numero divisibile per 2 è divisibile anche per 4} \vf
		\item {un numero non divisibile per 9 può essere divisibile per 3} \vf
		\item {un numero che termina con 0 è divisibile per 4} \vf
		\item {un numero non divisibile per 5 può essere divisibile per 10} \vf
		\item {un numero divisibile per 3 e per 5 è divisibile anche per 15} \vf
		\item {un numero divisibile per 3 e per 6 è divisibile anche per 18} \vf
		\item {un numero divisibile per 4 e per 9 è divisibile anche per 6} \vf
	\end{itemize}
	}


	\item {Eseguendo il calcolo a mente, scomporre i seguenti numeri: \\ \\ \textit{Esempio: }\(180 = 18 \times 10 = 2 \times 9 \times 2 \times 5 = 2^2 \times 3^2 	\times 5\)
		\begin{itemize}
			\item 56 =
			\item 132 =
			\item 280 =
			\item 150 =
			\item 120 =
			\item 75 =
			\item 88 =
		\end{itemize}
	}

	\item {Calcolare, mediante scomposizioni in fattori primi, M.C.D. e m.c.m. dei seguenti gruppi di numeri:
		\begin{itemize}
			\item 45, 18, 6, 15
			\item 54, 36, 24, 18
		\end{itemize}
	}

	\item {Risolvi i seguenti problemi:
		\begin{enumerate}[a.]
			\item {In un saponificio si produce sapone da bucato in pezzi da 220g, 250g e 350g ciascuno. Si vogliono confezionare questi pezzi in casse tutte dello stesso peso e contenenti ciascuna pezzi di sapone tutti uguali. Quale dovrà essere il peso minimo di ogni cassa?} \result{38,5}
			\item {Tre motociclisti percorrono nello stesso senso un circuito impiegando rispettivamente 14, 16 e 20 secondi per compiere un singolo giro. Se sono partiti insieme dal traguardo, quanti giri dovrà percorrere il primo motociclista prima di transitare dal traguardo contemporaneamente agli altri due?} \result{40}
		\end{enumerate}
	}

	\item {Ordinamento di frazioni: si procede riducendo allo stesso denominatore e confrontando i numeratori. \\ \\ \textit{Esempio: }Ordiniamo \[\frac{2}{15}, \frac{7}{12}, \frac{13}{40}, \frac{5}{16}\] \\ \\ Calcolando il m.c.m. tra i denominatori troviamo 240, perciò riscriviamo le frazioni come: \\ \[\frac{2}{15} = \frac{32}{240}\] \\ \[\frac{7}{12} = \frac{140}{240}\] \\ \[\frac{13}{40} = \frac{78}{240}\] \\ \[\frac{5}{16} = \frac{75}{240}\]\\\\Evinciamo perciò che: \[\frac{2}{15} < \frac{5}{16} < \frac{13}{40} < \frac{7}{12}\] \\ Mettere in ordine i seguenti gruppi di frazioni: \[\frac{13}{20},\frac{11}{12},\frac{10}{21},\frac12\] \[\frac92,2,\frac{11}{6},\frac73\]
	}

	\item {Risolvi le seguenti espressioni:}	
	
		\[a.\quad \left( 1-\frac23+\frac14 \right)^2\times\frac{48}{35}-\left(\frac35\right)^2\times\frac5{21}-\left(\frac12\right)^3:\frac7{12}\qquad\qquad\qquad\left[\frac16\right] \]
		
		\[b.\quad\frac1{20}\times\left[\left(2+\frac13\right)^2\times\frac37-1\right]^2+\left(a+\frac23-\frac32\right)^2-\frac1{30}\qquad\qquad\qquad\left[\frac1{12}\right]\]
		
		\[c.\quad\frac3{10}+\frac{33}{40}:\Bigg\{\frac3{10}+\frac57\times\frac{16}{35}\times\left[\left(\frac76-\frac34\right)^2:\frac5{36}-\frac38\right]^2\Bigg\}\qquad\qquad\qquad\left[\frac95\right]\]
		
		\[d.\quad\Bigg\{\left[\left(\frac52\right)^3-\left(\frac74-\frac56\right)\times\frac{10}{33}\times\left(\frac32\right)^2\frac56\right]\times\frac3{17}-\frac74\Bigg\}\times\left(\frac23\right)^2\qquad\qquad\qquad\left[\frac3{16}\right]\]
		
		\[e.\quad\Bigg\{\left[\left(\frac7{58}+\frac4{87}-\frac13:2\right)^3:\frac3{13}+\frac4{15}\right]^2:\left(\frac35\right)^2-\left(\frac23\right)^4\Bigg\}:\frac{19}{17}+1\qquad\qquad\qquad\left[1\right]\]
		
		\[f.\quad\frac{\frac7{12}-\frac{36}{25}\times\left(\frac56-\frac58\right)}{\left(\frac7{26}-\frac2{39}\right):\left(\frac5{28}\times\frac7{13}\right)}\qquad\qquad\qquad\left[\frac18\right]\]
		
		\[g.\quad\frac{\frac{39}{40}:\left(\frac4{15}:\frac6{35}-\frac56\right)}{\frac{83}{120}+\frac{17}{20}\times\left(\frac{21}{34}-\frac{19}{51}\right)}\qquad\qquad\qquad\left[\frac32\right]\]
		
		
		\item Calcolate il valore delle seguenti espressioni dopo aver trasformato in frazione i numeri decimali e periodici:
		
		\begin{enumerate}[a.]
			\item \(\quad(3,2\times1,4-1,18):0,6-0,5^2\times(2,2-0,2\times3^2)\) \result{\(\frac{27}5\)}
			
			\item \(\quad0,02\times[(4,3-3^2\times0,3)^2:0,2^4-2^5:0,04]-15\) \result{1}
			
			\item \(\quad0,4\overline{6}\times0,\overline{45}+0,75:3,\overline6\) \result{\(	\frac5{12}\)}
			
			\item \(\quad30-12,75:(0,0\overline5+0,41\overline6\) \result{3}
		\end{enumerate}
		
		\item Eseguire le seguenti addizioni mediante le opportune equivalenze:
		\begin{enumerate}[a.]
			\item 123 m + 432 cm + 125 dm = ................. m
			\item 27,89 dm\(^2\) + 0,37 m\(^2\) + 0,0038 km\(^2\) = .................
			\item 23 m\(^3\) + 2250 dm\(^3\) + 0,132 dam\(^3\) = ................. dm\(^3\)
			\item 73,8 dal + 0,27 dl + 0,73 l = ................. cl
			\item 12,5 hg + 32,7 kg + 1,023 q = ................. kg
		\end{enumerate}
		
		\item Un pilota ha percorso 5 giri di una pista in 13 minuti e 10 secondi. Quanto tempo impiegherà per percorrere 13 giri mantenendo sempre la stessa velocità media? \result{34 minuti e 14 secondi}
		
		\item Un orologio in 3 giorni ha ritardato di 7 minuti e 21 secondi. Quanto sarà in ritardo tra altri 5 giorni? \result{19 minuti e 36 secondi}
		
		\item Risolvere le seguenti proporzioni:
		\begin{enumerate}[a.]
			\item 14 : x = 7 : 5
			\item 42 : 30 = x : 20
			\item x : 20 = 24 : 30
			\item 10 : 15 = x : 9
		\end{enumerate}
		
		\item Risolvete le seguenti proporzioni applicando la proprietà del comporre e dello scomporre ed, eventualmente, le proprietà dell'invertire e del permutare
		\begin{enumerate}[a.]
			\item (x+5) : x = 22 : 12 \result{6}
			\item (28 - x) : x = 15 : 6 \result{8}
			\item (25+x) : 21 = x : 6 \result{10}
			\item 38 : 10 = (120-x) : x \result{25}
			\item (x+\(\frac38\)) : x = (x+\(\frac23\)) : \(\frac23\) \result{\(\frac12\)}
		\end{enumerate}
		
		
		\item Ricavate mediante le proprietà del comporre e dello scomporre i valori della x e della y dalle seguenti proporzioni:
		
		\begin{enumerate}[a.]
			\item x : y = 9 : 11 \quad sapendo che \quad x + y = 260
			\item x : y = 3 : 11 \quad sapendo che \quad x + y = 182
			\item x : y = 19 : 14 \quad sapendo che \quad x - y = 35
		\end{enumerate}
		
		\item Un negoziante acquista della merce a 235,00 \euro \, e la rivende a 290,00 \euro. Qual è il suo guadagno percentuale? \result{23,4\%}
		
		\item In un compito in classe di matematica:
		\begin{itemize}
		 	\item 1 alunno ha meritato 9
		 	\item 2 alunni hanno meritato 8
		 	\item 2 alunni hanno meritato 7
		 	\item 10 alunni hanno meritato 6
		 	\item 7 alunni hanno meritato 5
		 	\item 1 alunno ha meritato 4
		 	\item 2 alunni hanno meritato 3
		\end{itemize}
		
		Qual è la percentuale degli alunni che hanno meritato la sufficienza? \result{60\%}
		
		\item Una campana di bronzo è stata fabbricata fondendo dello stagno con 2,24 q di rame. Se la massa del rame è il 32\% della massa della campana, quanto stagno è occorso per la fusione? \result{4,72 q}
		
		\item Una puleggia compie 690 giri ogni 12 minuti; quanti giri compirà in 26 minuti ruotando sempre alla stessa velocità? \result{1495}
		
		\item Due tubi di ferro della stessa sezione sono lunghi rispettivamente 1,05 m e 1,55 m. Se il primo ha massa 7,56 kg a quanto ammonta la massa del secondo? \result{11,16 kg}
		
		\item Risolvere le seguenti espressioni applicando, ove possibile, le proprietà delle potenze:
		
		\[a.\quad\left(\frac14+\frac1{16}\right)+\left(-2+\frac34\right)^2:\frac54-\left(-1+\frac34-\frac12\right)^2\qquad\qquad\qquad\left[1\right]\]
		
		\[b.\quad\left[(-2)^2+\frac13\cdot\left(-\frac12\right)^4\cdot(-2)^5\right]:\left[\left(-\frac12\right)^2-\frac12\cdot(-2)\right]\qquad\qquad\qquad\left[\frac83\right]\]
		
		\[c.\quad\left[\left(-\frac12\right)^5:\left(-\frac12\right)^4-\frac12\right]^3:\left[\left(1+\frac12\right)\cdot\left(1-\frac12\right)\right]^2\qquad\qquad\qquad\left[-\frac{16}9\right]\]
		
		\[d.\quad\left(-\frac34+\frac12\right)^2:\left(\frac54-2\right)^2+\left(\frac54-1\right)^2\cdot\left(\frac13+5\right)-\left(-\frac12\right)\cdot\left(-2+\frac43\right)^2-1\qquad\qquad\qquad\left[-\frac13\right]\]
		
		\[e.\quad\left(2-\frac95\right)+\left[\left(\frac45-\frac3{10}\right)^3\cdot\left(\frac{27}{20}:\frac14-5\right)^3+\left(-1+\frac45\right)^2\right]:\left(-2+\frac85\right)^2+\frac13\qquad\qquad\qquad\left[\frac56\right]\]
		
		\[f.\quad\left(1-\frac25\right)^2:\left[1-\left(-\frac12\right)^5:\left(-\frac12\right)^4\right]\cdot\left(1+\frac78\right)\qquad\qquad\qquad\left[\frac9{20}\right]\]
		
		\[g.\quad\frac{\left(\frac25+\frac14\right):\left(\frac1{10}-\frac34\right)}{2-\left(-\frac56\right):\left(\frac59-\frac{11}6+\frac7{12}\right)}\qquad\qquad\qquad\left[-\frac54\right]\]
		
		
\end{enumerate}

\vfill

\begin{center}
	Realizzato utilizzando \LaTeX \\ Guarda il codice, commenta e apporta migliorie su GitHub \includegraphics[height=0.25cm]{git} \\ \parskip=25pt
\tiny{https://github.com/AlexF1789/compitiPrime}
\end{center}

\end{document}
